%%%%%%%%%%%%%%%%%%%%%%%%%%%%%%%%%%%%%%%%%
% University/School Laboratory Report
% LaTeX Template
% Version 3.1 (25/3/14)
%
% This template has been downloaded from:
% http://www.LaTeXTemplates.com
%
% Original author:
% Linux and Unix Users Group at Virginia Tech Wiki 
% (https://vtluug.org/wiki/Example_LaTeX_chem_lab_report)
%
% License:
% CC BY-NC-SA 3.0 (http://creativecommons.org/licenses/by-nc-sa/3.0/)
%
%%%%%%%%%%%%%%%%%%%%%%%%%%%%%%%%%%%%%%%%%

%----------------------------------------------------------------------------------------
%	PACKAGES AND DOCUMENT CONFIGURATIONS
%----------------------------------------------------------------------------------------

\documentclass{article}

\usepackage[version=3]{mhchem} % Package for chemical equation typesetting
\usepackage{siunitx} % Provides the \SI{}{} and \si{} command for typesetting SI units
\usepackage{graphicx} % Required for the inclusion of images
\usepackage{natbib} % Required to change bibliography style to APA
\usepackage{amsmath} % Required for some math elements 
\usepackage{float}

\setlength\parindent{0pt} % Removes all indentation from paragraphs

\renewcommand{\labelenumi}{\alph{enumi}.} % Make numbering in the enumerate environment by letter rather than number (e.g. section 6)

%\usepackage{times} % Uncomment to use the Times New Roman font

%----------------------------------------------------------------------------------------
%	DOCUMENT INFORMATION
%----------------------------------------------------------------------------------------

\title{Probabilistic Graphical Models\\Koller\\The Bayesian Networks} % Title

\author{Ahmad \textsc{Asadi}} % Author name

\date{\today} % Date for the report

\begin{document}

\maketitle % Insert the title, author and date

\begin{center}
\begin{tabular}{l r}
Date Performed: & March 12, 2016 \\ % Date the experiment was performed
%Partners: & James Smith \\ % Partner names
%& Mary Smith \\
Instructor: & Dr. Nickabadi % Instructor/supervisor
\end{tabular}
\end{center}

% If you wish to include an abstract, uncomment the lines below
% \begin{abstract}
% Abstract text
% \end{abstract}

%----------------------------------------------------------------------------------------
%	SECTION 1
%----------------------------------------------------------------------------------------

\section{Basics}
\begin{description}
\item[Why to use independency] Using random variable's independencies is effective in reducing the size of joint probability distribution over them.
\item[Chain Rule] The chain rule of conditional probabilities:
\begin{equation}
	P(X_1, X_2, \cdots, X_n) = P(x_1)\cdot P(X_2|X_1)\cdot P(X_3|X_1, X_2) \cdots P(X_n|X_{n-1}, X_{n-2}, \cdots, X_1) 
\end{equation}
\item[Factorization] Splitting the overall joint probability distribution into several conditional probability distributions (CPDs) where multiplications of CPDs reconstructs joint probability distribution.

\end{description}

\section{Naive/Idiot Bayes}
The model includes a set of classes C, some number of features $X_1,\cdots,X_n$ and assumes that \textit{features are conditionally independent given the instance's class}:\\
\begin{equation}
P(C,X_1,\cdots,X_n) = P(C)\cdot \Pi_{i=1}^n P(X_i|C)
\end{equation}
Factors in this model:
\begin{enumerate}
\item A prior distribution $P(C)$
\item A set of CPDs $P(X_j|C)$
\end{enumerate}
\section{Bayesian Networks}
\begin{description}
\item[DAG Representation]
The core element of representation in bayesian networks is a \textit{directed acyclic graph (DAG)}, in which random variables are nodes and edges correspond to direct influence of random variables on each other.\\
This DAG can be viewed as:\\
	\begin{enumerate}
	\item A data structure that provides the skeleton for representing a joint distribution compactly in a factorized way.
	\item A compact representation for a set of conditional independence assumptions about a distribution.
	\end{enumerate}
	
\item[Local Probability Models] A set of local probability
models that represent the nature of the dependence of each variable on its parents, containing probability distributions of single random variables along with CPDs in model.

\item[Reasoning Patterns] A joint distribution $P_B$ specifies the probability $P_B (Y = y | E = e)$ of any event $y$ given any observations $e$. We condition the joint distribution on the event $E = e$ by eliminating the entries in the joint inconsistent with our observation $e$, and renormalizing the resulting entries to sum to 1; we compute the probability of the event $y$ by summing the probabilities of all of the entries in the resulting posterior distribution that are consistent with $y$.
\begin{description}
	\item[Causal Reasoning] is a top-down flow of probability computations in DAG from causes to symptoms.
	\item[Evidential Reasoning] is a bottom-up flow of probability computations in DAG from a symptom to its causes which yields the probability of each cause to be happened.
	\item[Intercausal Reasoning] is a cross flow of probability computations in DAG from a cause to another cause passing collidors in the path.
\end{description}

\item[Basic Independencies in Bayesian Networks]
\end{description}


%----------------------------------------------------------------------------------------
%	BIBLIOGRAPHY
%----------------------------------------------------------------------------------------

\bibliographystyle{apalike}

\bibliography{sample}

%----------------------------------------------------------------------------------------


\end{document}